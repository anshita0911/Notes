\documentclass[10pt,british]{article}
\usepackage[T1]{fontenc}
\usepackage[latin9]{luainputenc}
\usepackage[a4paper]{geometry}
\geometry{verbose,tmargin=2cm,bmargin=2cm,lmargin=2cm,rmargin=2cm}
\usepackage{fancyhdr}
\pagestyle{fancy}
\setcounter{tocdepth}{2}
\usepackage{babel}
\usepackage{calc}
\usepackage{amsmath}
\usepackage{setspace}
\usepackage{microtype}
\onehalfspacing
\usepackage[unicode=true,pdfusetitle,
 bookmarks=true,bookmarksnumbered=true,bookmarksopen=false,
 breaklinks=false,pdfborder={0 0 0},pdfborderstyle={},backref=section,colorlinks=false]
 {hyperref}

\makeatletter

\providecommand{\tabularnewline}{\\}
\@ifundefined{date}{}{\date{}}
\makeatletter
\@addtoreset{section}{part}
\makeatother
\usepackage[nobottomtitles*]{titlesec}
\fancyhead[RO]{\textsl{TABLE OF CONTENTS}}
\fancyhead{}
\fancyhead[RE,LO]{\leftmark}

\AtBeginDocument{
  \def\labelitemi{\(\star\)}
}

\makeatother

\begin{document}
\title{\textbf{\Huge{}CRYPTOGRAPHY}}
\author{AGNI DATTA}
\maketitle
\begin{center}
\rule[0.5ex]{0.5\columnwidth}{0.75pt}
\par\end{center}
\begin{quotation}
\begin{center}
\textbf{`We can only see a short distance ahead, but we can see plenty
there that needs to be done'}
\par\end{center}
\begin{center}
Alan Turing
\par\end{center}

\end{quotation}
\medskip{}

\begin{abstract}
{\normalsize{}Cryptography --- the science of secret writing ---
is an ancient art; the first documented use of cryptography in writing
dates back to circa 1900 B.C. when an Egyptian scribe used non-standard
hieroglyphs in an inscription. Some experts argue that cryptography
appeared spontaneously sometime after writing was invented, with applications
ranging from diplomatic missives to war-time battle plans. It is no
surprise, then, that new forms of cryptography came soon after the
widespread development of computer communications. In data and telecommunications,
cryptography is necessary when communicating over any untrusted medium,
which includes just about any network, particularly the Internet.}{\normalsize\par}

\pagebreak{}
\end{abstract}
\tableofcontents{}

\pagebreak{}

\section{Glossary}
\begin{center}
\rule[0.5ex]{450bp}{0.75pt}
\par\end{center}

\subsubsection{Cryptography}

The transformed message cipher an algorithm for transforming an intelligible
message into one that is unintelligible by transposition and/or substitution
methods the art or science encompassing the principles and methods
of transforming an intelligible message into one that is unintelligible,
and then re-transforming that message back to its original form.

\subsubsection{Plaintext}

The original intelligible message.

\subsubsection{Ciphertext}

The transformed message cipher an algorithm for transforming an intelligible
message into one that is unintelligible by transposition and/or substitution
methods.

\noindent\fbox{\begin{minipage}[t]{1\columnwidth - 2\fboxsep - 2\fboxrule}%
\begin{center}
$ P\:=\:\textbf{plaintext}\:\textbf{and}\:C\:=\:\textbf{ciphertext} $
\par\end{center}%
\end{minipage}}

\subsubsection{Key}

Some critical information used by the cipher, known only to the sender
and receiver.

\subsubsection{Encipher(Encode)}

The process of converting plaintext to ciphertext using a cipher and
a key.

\subsubsection{Decipher(Decode)}

The process of converting ciphertext back into plaintext using a cipher
and a key.

\subsubsection{Cryptanalysis}

The study of principles and methods of transforming an unintelligible
message back into an intelligible message without knowledge of the
key. Also called codebreaking.

\subsubsection{Cryptology}

The sum of both cryptography and cryptanalysis.

\subsubsection{Code}

An algorithm for transforming an intelligible message into an unintelligible
one using a code-book.

\pagebreak{}

\section{Introduction To Cryptography}
\begin{center}
\rule[0.5ex]{450bp}{0.75pt}
\par\end{center}

\subsection{Cryptography}

Cryptography, or cryptology, is the practice and study of techniques
for secure communication in the presence of third parties called adversaries.
More generally, cryptography is about constructing and analysing protocols
that prevent third parties or the public from reading private messages;
various aspects in information security such as data confidentiality,
data integrity, authentication, and non-repudiation are central to
modern cryptography. Modern cryptography exists at the intersection
of the disciplines of mathematics, computer science, electrical engineering,
communication science, and physics. Applications of cryptography include
electronic commerce, chip-based payment cards, digital currencies,
computer passwords, and military communications. There are five primary
functions of cryptography:
\begin{center}
\begin{tabular}{|l||l|}
\hline 
\textbf{\small{}Privacy/Confidentiality} & {\small{}Ensuring that no one can read the message except the intended
receiver.}\tabularnewline
\hline 
\textbf{\small{}Authentication} & {\small{}The process of proving one's identity.}\tabularnewline
\hline 
\textbf{\small{}Integrity} & {\small{}Assuring the receiver that the received message has not been
altered in any way.}\tabularnewline
\hline 
\textbf{\small{}Non-repudiation} & {\small{}A mechanism to prove that the sender really.}\tabularnewline
\hline 
\textbf{\small{}Key exchange} & {\small{}The method by which crypto keys are shared between sender
and receiver.}\tabularnewline
\hline 
\end{tabular}
\par\end{center}

\medskip{}

In cryptography, we start with the unencrypted data, referred to as
plaintext. Plaintext is encrypted into ciphertext, which will in turn
(usually) be decrypted back into usable plaintext. The encryption
and decryption is based upon the type of cryptography scheme being
employed and some form of key. For those who like formulas, this process
is sometimes written as:

$$ C = E_k(P) $$
$$ P = D_k(C) $$

where, 

$P$ = plaintext,

$C$ = ciphertext,

$E$ = the encryption method,

$D$ = the decryption method,

$k$ = the key.

Given this, there are other functions that might be supported by crypto
and other terms that one might hear:
\begin{itemize}
\item \textbf{Forward Secrecy (aka Perfect Forward Secrecy):} This feature
protects past encrypted sessions from compromise even if the server
holding the messages is compromised. This is accomplished by creating
a different key for every session so that compromise of a single key
does not threaten the entirely of the communications. 
\item \textbf{Perfect Security:} A system that is unbreakable and where
the ciphertext conveys no information about the plaintext or the key.
To achieve perfect security, the key has to be at least as long as
the plaintext, making analysis and even brute-force attacks impossible.
One-time pads are an example of such a system.
\item \textbf{Deniable Authentication (aka Message Repudiation): }A method
whereby participants in an exchange of messages can be assured in
the authenticity of the messages but in such a way that senders can
later plausibly deny their participation to a third-party.
\end{itemize}
In many of the descriptions below, two communicating parties will
be referred to as Alice and Bob; this is the common nomenclature in
the crypto field and literature to make it easier to identify the
communicating parties. If there is a third and fourth party to the
communication, they will be referred to as Carol and Dave, respectively.
A malicious party is referred to as Mallory, an eavesdropper as Eve,
and a trusted third party as Trent.

Finally, cryptography is most closely associated with the development
and creation of the mathematical algorithms used to encrypt and decrypt
messages, whereas cryptanalysis is the science of analysing and breaking
encryption schemes. Cryptology is the umbrella term referring to the
broad study of secret writing, and encompasses both cryptography and
cryptanalysis.

\subsection{Cryptanalysis}

\paragraph{\textmd{Cryptanalysis is the study of analysing information systems
in order to study the hidden aspects of the systems. Cryptanalysis
is used to breach cryptographic security systems and gain access to
the contents of encrypted messages, even if the cryptographic key
is unknown. In addition to mathematical analysis of cryptographic
algorithms, cryptanalysis includes the study of side-channel attacks
that do not target weaknesses in the cryptographic algorithms themselves,
but instead exploit weaknesses in their implementation.}}

\subsection{Private-Key Cryptography}

\paragraph{\textmd{Symmetric-key algorithms are algorithms for cryptography
that use the same cryptographic keys for both the encryption of plaintext
and the decryption of ciphertext. The keys may be identical, or there
may be a simple transformation to go between the two keys. The keys,
in practice, represent a shared secret between two or more parties
that can be used to maintain a private information link. The requirement
that both parties have access to the secret key is one of the main
drawbacks of symmetric-key encryption, in comparison to public-key
encryption (also known as asymmetric-key encryption).}}
\begin{center}
$$ P\xrightarrow{\:\textbf{private key}\:}C\xrightarrow{\:\textbf{private key}\:}P $$
\par\end{center}

\subsection{Public-Key Cryptography}

\paragraph{\textmd{Public-key cryptography, or asymmetric cryptography, is a
cryptographic system that uses pairs of keys: public keys (which may
be known to others), and private keys (which may never be known by
any except the owner). The generation of such key pairs depends on
cryptographic algorithms which are based on mathematical problems
termed one-way functions. Effective security requires keeping the
private key private; the public key can be openly distributed without
compromising security.}}

\paragraph{\textmd{In such a system, any person can encrypt a message using
the intended receiver's public key, but that encrypted message can
only be decrypted with the receiver's private key. This allows, for
instance, a server program to generate a cryptographic key intended
for a suitable symmetric-key cryptography, then to use a client's
openly-shared public key to encrypt that newly generated symmetric
key. The server can then send this encrypted symmetric key over an
insecure channel to the client; only the client can decrypt it using
the client's private key (which pairs with the public key used by
the server to encrypt the message). With the client and server both
having the same symmetric key, they can safely use symmetric key encryption
(likely much faster) to communicate over otherwise-insecure channels.
This scheme has the advantage of not having to manually pre-share
symmetric keys (a fundamentally difficult problem) while gaining the
higher data throughput advantage of symmetric-key cryptography.}}

\paragraph{\textmd{With public-key cryptography, robust authentication is also
possible. A sender can combine a message with a private key to create
a short digital signature on the message. Anyone with the sender's
corresponding public key can combine that message with a claimed digital
signature; if the signature matches the message, the origin of the
message is verified (i.e., it must have been made by the owner of
the corresponding private key).}}

\paragraph{\textmd{Public key algorithms are fundamental security primitives
in modern cryptosystems, including applications and protocols which
offer assurance of the confidentiality, authenticity and non-availability
of electronic communications and data storage. They underpin numerous
Internet standards, such as Transport Layer Security (TLS), S/MIME,
PGP, and GPG. Some public key algorithms provide key distribution
and secrecy (e.g., Diffie--Hellman key exchange), some provide digital
signatures (e.g., Digital Signature Algorithm), and some provide both
(e.g., RSA). Compared to symmetric encryption, asymmetric encryption
is rather slower than good symmetric encryption, too slow for many
purposes. Today's cryptosystems (such as TLS, Secure Shell) use both
symmetric encryption and asymmetric encryption.}}
\begin{center}
$$ P\xrightarrow{\:\textbf{public key}\:}C\xrightarrow{\:\textbf{private key}\:}P $$
\par\end{center}

\subsection{Hash Function}

\paragraph{\textmd{A hash function is any function that can be used to map data
of arbitrary size to fixed-size values. The values returned by a hash
function are called hash values, hash codes, digests, or simply hashes.
The values are usually used to index a fixed-size table called a hash
table. Use of a hash function to index a hash table is called hashing
or scatter storage addressing.}}

\paragraph{\textmd{Hash functions and their associated hash tables are used
in data storage and retrieval applications to access data in a small
and nearly constant time per retrieval, and require an amount of storage
space only fractionally greater than the total space required for
the data or records themselves. Hashing is a computationally and storage
space efficient form of data access which avoids the non-linear access
time of ordered and unordered lists and structured trees, and the
often exponential storage requirements of direct access of state spaces
of large or variable-length keys.}}

\paragraph{\textmd{Use of hash functions relies on statistical properties of
key and function interaction: worst case behaviour is intolerably
bad with a vanishingly small probability, and average case behaviour
can be nearly optimal (minimal collision).}}

\paragraph{\textmd{Hash functions are related to (and often confused with) checksums,
check digits, fingerprints, lossy compression, randomization functions,
error-correcting codes, and ciphers. Although the concepts overlap
to some extent, each one has its own uses and requirements and is
designed and optimized differently. The hash functions differ from
the concepts numbered mainly in terms of data integrity.}}
\begin{center}
$$ P\xrightarrow{\:\textbf{hash function}\:}C $$
\par\end{center}
\end{document}
